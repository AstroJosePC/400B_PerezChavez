
%% Beginning of file 'sample63.tex'
%%
%% Modified 2019 June
%%
%% This is a sample manuscript marked up using the
%% AASTeX v6.3 LaTeX 2e macros.
%%
%% AASTeX is now based on Alexey Vikhlinin's emulateapj.cls 
%% (Copyright 2000-2015).  See the classfile for details.

%% AASTeX requires revtex4-1.cls (http://publish.aps.org/revtex4/) and
%% other external packages (latexsym, graphicx, amssymb, longtable, and epsf).
%% All of these external packages should already be present in the modern TeX 
%% distributions.  If not they can also be obtained at www.ctan.org.

%% The first piece of markup in an AASTeX v6.x document is the \documentclass
%% command. LaTeX will ignore any data that comes before this command. The 
%% documentclass can take an optional argument to modify the output style.
%% The command below calls the preprint style which will produce a tightly 
%% typeset, one-column, single-spaced document.  It is the default and thus
%% does not need to be explicitly stated.
%%
%%
%% using aastex version 6.3
\documentclass{aastex63}

%% The default is a single spaced, 10 point font, single spaced article.
%% There are 5 other style options available via an optional argument. They
%% can be invoked like this:
%%
%% \documentclass[arguments]{aastex63}
%% 
%% where the layout options are:
%%
%%  twocolumn   : two text columns, 10 point font, single spaced article.
%%                This is the most compact and represent the final published
%%                derived PDF copy of the accepted manuscript from the publisher
%%  manuscript  : one text column, 12 point font, double spaced article.
%%  preprint    : one text column, 12 point font, single spaced article.  
%%  preprint2   : two text columns, 12 point font, single spaced article.
%%  modern      : a stylish, single text column, 12 point font, article with
%% 		  wider left and right margins. This uses the Daniel
%% 		  Foreman-Mackey and David Hogg design.
%%  RNAAS       : Preferred style for Research Notes which are by design 
%%                lacking an abstract and brief. DO NOT use \begin{abstract}
%%                and \end{abstract} with this style.
%%
%% Note that you can submit to the AAS Journals in any of these 6 styles.
%%
%% There are other optional arguments one can invoke to allow other stylistic
%% actions. The available options are:
%%
%%   astrosymb    : Loads Astrosymb font and define \astrocommands. 
%%   tighten      : Makes baselineskip slightly smaller, only works with 
%%                  the twocolumn substyle.
%%   times        : uses times font instead of the default
%%   linenumbers  : turn on lineno package.
%%   trackchanges : required to see the revision mark up and print its output
%%   longauthor   : Do not use the more compressed footnote style (default) for 
%%                  the author/collaboration/affiliations. Instead print all
%%                  affiliation information after each name. Creates a much 
%%                  longer author list but may be desirable for short 
%%                  author papers.
%% twocolappendix : make 2 column appendix.
%%   anonymous    : Do not show the authors, affiliations and acknowledgments 
%%                  for dual anonymous review.
%%
%% these can be used in any combination, e.g.
%%
%% \documentclass[twocolumn,linenumbers,trackchanges]{aastex63}
%%
%% AASTeX v6.* now includes \hyperref support. While we have built in specific
%% defaults into the classfile you can manually override them with the
%% \hypersetup command. For example,
%%
%% \hypersetup{linkcolor=red,citecolor=green,filecolor=cyan,urlcolor=magenta}
%%
%% will change the color of the internal links to red, the links to the
%% bibliography to green, the file links to cyan, and the external links to
%% magenta. Additional information on \hyperref options can be found here:
%% https://www.tug.org/applications/hyperref/manual.html#x1-40003
%%
%% Note that in v6.3 "bookmarks" has been changed to "true" in hyperref
%% to improve the accessibility of the compiled pdf file.
%%
%% If you want to create your own macros, you can do so
%% using \newcommand. Your macros should appear before
%% the \begin{document} command.
%%
\newcommand{\vdag}{(v)^\dagger}
\newcommand\aastex{AAS\TeX}
\newcommand\latex{La\TeX}
\usepackage{listings}

%% Reintroduced the \received and \accepted commands from AASTeX v5.2
%% \received{June 1, 2019}
%% \revised{January 10, 2019}
%% \accepted{\today}
%% Command to document which AAS Journal the manuscript was submitted to.
%% Adds "Submitted to " the argument.
%% \submitjournal{AJ}

%% For manuscript that include authors in collaborations, AASTeX v6.3
%% builds on the \collaboration command to allow greater freedom to 
%% keep the traditional author+affiliation information but only show
%% subsets. The \collaboration command now must appear AFTER the group
%% of authors in the collaboration and it takes TWO arguments. The last
%% is still the collaboration identifier. The text given in this
%% argument is what will be shown in the manuscript. The first argument
%% is the number of author above the \collaboration command to show with
%% the collaboration text. If there are authors that are not part of any
%% collaboration the \nocollaboration command is used. This command takes
%% one argument which is also the number of authors above to show. A
%% dashed line is shown to indicate no collaboration. This example manuscript
%% shows how these commands work to display specific set of authors 
%% on the front page.
%%
%% For manuscript without any need to use \collaboration the 
%% \AuthorCollaborationLimit command from v6.2 can still be used to 
%% show a subset of authors.
%
%\AuthorCollaborationLimit=2
%
%% will only show Schwarz & Muench on the front page of the manuscript
%% (assuming the \collaboration and \nocollaboration commands are
%% commented out).
%%
%% Note that all of the author will be shown in the published article.
%% This feature is meant to be used prior to acceptance to make the
%% front end of a long author article more manageable. Please do not use
%% this functionality for manuscripts with less than 20 authors. Conversely,
%% please do use this when the number of authors exceeds 40.
%%
%% Use \allauthors at the manuscript end to show the full author list.
%% This command should only be used with \AuthorCollaborationLimit is used.

%% The following command can be used to set the latex table counters.  It
%% is needed in this document because it uses a mix of latex tabular and
%% AASTeX deluxetables.  In general it should not be needed.
%\setcounter{table}{1}

%%%%%%%%%%%%%%%%%%%%%%%%%%%%%%%%%%%%%%%%%%%%%%%%%%%%%%%%%%%%%%%%%%%%%%%%%%%%%%%%
%%
%% The following section outlines numerous optional output that
%% can be displayed in the front matter or as running meta-data.
%%
%% If you wish, you may supply running head information, although
%% this information may be modified by the editorial offices.
\shorttitle{ASTR400B Project}
\shortauthors{José Pérez}
%%
%% You can add a light gray and diagonal water-mark to the first page 
%% with this command:
%% \watermark{text}
%% where "text", e.g. DRAFT, is the text to appear.  If the text is 
%% long you can control the water-mark size with:
%% \setwatermarkfontsize{dimension}
%% where dimension is any recognized LaTeX dimension, e.g. pt, in, etc.
%%
%%%%%%%%%%%%%%%%%%%%%%%%%%%%%%%%%%%%%%%%%%%%%%%%%%%%%%%%%%%%%%%%%%%%%%%%%%%%%%%%
\graphicspath{{./}{figures/}}
%% This is the end of the preamble.  Indicate the beginning of the
%% manuscript itself with \begin{document}.

\begin{document}

\title{Tidal Debris from M33: Stellar Streams of M33}

%% LaTeX will automatically break titles if they run longer than
%% one line. However, you may use \\ to force a line break if
%% you desire. In v6.3 you can include a footnote in the title.

%% A significant change from earlier AASTEX versions is in the structure for 
%% calling author and affiliations. The change was necessary to implement 
%% auto-indexing of affiliations which prior was a manual process that could 
%% easily be tedious in large author manuscripts.
%%
%% The \author command is the same as before except it now takes an optional
%% argument which is the 16 digit ORCID. The syntax is:
%% \author[xxxx-xxxx-xxxx-xxxx]{Author Name}
%%
%% This will hyperlink the author name to the author's ORCID page. Note that
%% during compilation, LaTeX will do some limited checking of the format of
%% the ID to make sure it is valid. If the "orcid-ID.png" image file is 
%% present or in the LaTeX pathway, the OrcID icon will appear next to
%% the authors name.
%%
%% Use \affiliation for affiliation information. The old \affil is now aliased
%% to \affiliation. AASTeX v6.3 will automatically index these in the header.
%% When a duplicate is found its index will be the same as its previous entry.
%%
%% Note that \altaffilmark and \altaffiltext have been removed and thus 
%% can not be used to document secondary affiliations. If they are used latex
%% will issue a specific error message and quit. Please use multiple 
%% \affiliation calls for to document more than one affiliation.
%%
%% The new \altaffiliation can be used to indicate some secondary information
%% such as fellowships. This command produces a non-numeric footnote that is
%% set away from the numeric \affiliation footnotes.  NOTE that if an
%% \altaffiliation command is used it must come BEFORE the \affiliation call,
%% right after the \author command, in order to place the footnotes in
%% the proper location.
%%
%% Use \email to set provide email addresses. Each \email will appear on its
%% own line so you can put multiple email address in one \email call. A new
%% \correspondingauthor command is available in V6.3 to identify the
%% corresponding author of the manuscript. It is the author's responsibility
%% to make sure this name is also in the author list.
%%
%% While authors can be grouped inside the same \author and \affiliation
%% commands it is better to have a single author for each. This allows for
%% one to exploit all the new benefits and should make book-keeping easier.
%%
%% If done correctly the peer review system will be able to
%% automatically put the author and affiliation information from the manuscript
%% and save the corresponding author the trouble of entering it by hand.

\author{José Pérez Chávez}
\affiliation{University of Arizona}

%% Note that the \and command from previous versions of AASTeX is now
%% depreciated in this version as it is no longer necessary. AASTeX 
%% automatically takes care of all commas and "and"s between authors names.

%% AASTeX 6.3 has the new \collaboration and \nocollaboration commands to
%% provide the collaboration status of a group of authors. These commands 
%% can be used either before or after the list of corresponding authors. The
%% argument for \collaboration is the collaboration identifier. Authors are
%% encouraged to surround collaboration identifiers with ()s. The 
%% \nocollaboration command takes no argument and exists to indicate that
%% the nearby authors are not part of surrounding collaborations.

%% From the front matter, we move on to the body of the paper.
%% Sections are demarcated by \section and \subsection, respectively.
%% Observe the use of the LaTeX \label
%% command after the \subsection to give a symbolic KEY to the
%% subsection for cross-referencing in a \ref command.
%% You can use LaTeX's \ref and \label commands to keep track of
%% cross-references to sections, equations, tables, and figures.
%% That way, if you change the order of any elements, LaTeX will
%% automatically renumber them.
%%
%% We recommend that authors also use the natbib \citep
%% and \citet commands to identify citations.  The citations are
%% tied to the reference list via symbolic KEYs. The KEY corresponds
%% to the KEY in the \bibitem in the reference list below. 

\section{Introduction} \label{sec:intro}

Our Milky Way Galaxy is in a merging course with its neighbor, 
the Andromeda galaxy (M31). These two galaxies, and their respective
satellite galaxies form our Local Group. The term Local Group simply refers
to the fact that our galaxies gravitational forces are significant enough to
affect their evolution, and can't be ignored. With the new M31 proper motion
measurement we can predict the timing of the collision between the MW and
M31: $3.87^{+0.42}_{-0.32}$ Gyr \citep{2012ApJ...753....9V}. 

The merging process will cause dynamical structures within the galaxy that
tidally evolve over millions of years. The evolution of these 
structures can help us better understand galaxy mass distributions and the 
shapes of the black matter halos. Mergers simulations 
allow us to explore the evolution of said structures and study them. 
\citet{2007MNRAS.381..987C} investigates satellite galaxy evolution by 
performing high-resolution N-body, and find an intrinsic mass dependence
that provides additional leverage on both halo and progenitor satellite 
properties. 

In \citep{2012ApJ...753....9V}, scientist presented an entirely deterministic
Local Group evolution dataset as a result of recent Local Group phase-space 
measurements, and improvements thereof. The kinematic information of our own 
Local Group is essential to reproducing accurate evolution models of galaxies.
This is not the case for farther galaxies since extracting such data is much 
harder to get. While Milky way like N-body simulations are effective at telling
us about galaxy evolution, the realistic initial state of our Local Group can
probe important information that we are missing.

In this paper we will focus on the M33 streams that survive the MW-M31 merger.
Tidal streams act as past constraints on the progenitor orbits and allow us to
probe the parent galaxy's gravity, and hence its dark matter halo. In Andromeda, 
we know of roughly 4 stellar streams such as the Giant stellar 
stream, Andromeda NE, Tidal Stream Northwest, and Tidal Stream Southwest 
streams \citep{2010AAS...21535401G, 2006AAS...208.4803G}. \citet{2007MNRAS.381..987C, 2017MNRAS.464.2882A} have explored the distribution of tidal streams to assess galaxy morphology, and phase-space distribution.

\section{Proposal} \label{sec:proposal}

For my project I want to track M33 streams distribution that will be located 
in the MW-M31 merged halo. At the same time I want to assess how well the streams 
trace the orbital path of M33. I want to produce a qualitative timeline for the satellite galaxy disruption and stellar stream creation. Additionally, and potentially optionally, I want to assess the detection of dark matter subhalos.

\subsection{Tracking Streams\label{subsec:streams}}

A technique I can implement to track the stellar streams is to use the Jacobi
radius, and label those stars outside of it. Jacobi (or tidal) radius is the 
distance from the center of the satellite galaxy at which the external gravitation 
of the host galaxy has more influence over the stars in the satellite than does the
satellite itself. The initial M33 stars are already labeled, so I need to add 
a second layer to my pipeline in order to track those outside the tidal radius.
The stellar streams orientation is also important, so once each stream is labelled separately, I can calculate the morphology using basic techniques such as moments.

Another interesting feature of streams is that they can potentially be used to 
indirectly detect dark matter clumps. When a dark matter subhalo clump grazes a 
stellar stream, it could disrupt the stream and produce gaps as indicated in 
\citet{2018AAS...23121203P}\footnote{It seems like there is no source for this 
paper, but I found a video of her presentation:\url{https://youtu.be/HJJsIHrBBLQ}}.
However, as they discuss, I must track the effect of the galactic centers (MW/M31),
as it also generates underdensities in streams. I think I should keep this part of 
the project as optional since it may require the most time. I will need to spend
some time in finding an optimal way to find the underdensities in the halo to
find the subhalos. While \citet{2018AAS...23121203P} explores this process by
inserting artificial streams, I am interested in finding the impact from 
new realistic incoming streams from M33.

\subsection{Visualization\label{subsec:viz}}

I will use Blender software to produce a custom visualization of the streams.
Blender is the free and open source 3D creation suite. There are other similar
projects, but the advantage of using Blender is that it is open source, and as
such there is a lot of documentation. It provides a Python interface for all 
of its features. This allows me to import the Python code we have generated
in class and for assignments, create custom animations, run everything in 
a single pipeline.

The \citet{2012ApJ...753....9V} MW-M31-M33 merger system dataset has 800 
snapshots that will produce a 24 frame rate animation of approximately 100 
seconds assuming I can introduce three interpolated frames between each 
snapshot. Blender has a Python interface capable of performing said animation.

As of now, I have a working code that will render one frame at a time.
The results are shown in Figure \ref{fig:snapshots}. While this work is 
awesome, I need to come up with a way to automate the procedure such that I 
don't have to process each snapshot manually. Here is an example of two 
frames. The Blender Cycles engine is capable of rendering all ~42,500 
particles from the low resolution dataset with no problem. 

%% The "ht!" tells LaTeX to put the figure "here" first, at the "top" next
%% and to override the normal way of calculating a float position
\begin{figure}[h!]
\plottwo{render0.pdf}{render1.pdf}
\caption{The lefthand picture shows the initial  snapshot of the Milky Way, 
while the righthand picture shows the 50th snapshot of the simulation.
Both pictures were taken from a Z-axis perspective from the initial 
arbitrary reference frame.
\label{fig:snapshots}}
\end{figure}

In order to proceed with this method of visualization I need to learn how to 
move the scene camera, the one that produces the images above, using the 
Python interface to follow the galaxy, or certain components of it.
Blender also has a few features that should allow me to apply different colors
to different particles; which will be useful for visualizing the streams. Another
improvement is to attempt working with the full resolution dataset to improve
my results accuracy, and push the limits of this method.

\subsection{Expected Results\label{subsec:results}}

For my results I should expect, as \citet{2017MNRAS.464.2882A, 2007MNRAS.381..987C}
found, M33 being a relatively massive satellite galaxy won't produce efficient 
clues to its orbital path. This is the same for the tail orientation; it won't be a
great indicator for path. However, M33's stellar debris is expected to fall deep 
into the potential, which means any incoming subhalo remnant should dissipate fast 
\citep{2017MNRAS.464.2882A}. As for the stream hole-punching subhalo, I should expect the galactic centers to have a greater impact in stellar stream underdensities than any subhalo clump, especially for non-retrograde streams, and 

% \begin{figure*}
% \gridline{\fig{V2491_Cyg.pdf}{0.3\textwidth}{(a)}
%           \fig{HV_Cet.pdf}{0.3\textwidth}{(b)}
%           \fig{LMC_2009.pdf}{0.3\textwidth}{(c)}
%           }
% \gridline{\fig{RS_Oph.pdf}{0.3\textwidth}{(d)}
%           \fig{U_Sco.pdf}{0.3\textwidth}{(e)}
%           }
% \gridline{\fig{KT_Eri.pdf}{0.3\textwidth}{(f)}}
% \caption{Inverted pyramid figure of six individual files. The nova are
% (a) V2491 Cyg, (b) HV Cet, (c) LMC 2009, (d) RS Oph, (e) U Sco, and (f) 
% KT Eri. These individual figures are taken from \citet{2011ApJS..197...31S}.
% \label{fig:pyramid}}
% \end{figure*}



%% For this sample we use BibTeX plus aasjournals.bst to generate the
%% the bibliography. The sample63.bib file was populated from ADS. To
%% get the citations to show in the compiled file do the following:
%%
%% pdflatex sample63.tex
%% bibtext sample63
%% pdflatex sample63.tex
%% pdflatex sample63.tex

\bibliography{sample63}{}
\bibliographystyle{aasjournal}

%% This command is needed to show the entire author+affiliation list when
%% the collaboration and author truncation commands are used.  It has to
%% go at the end of the manuscript.
%\allauthors

%% Include this line if you are using the \added, \replaced, \deleted
%% commands to see a summary list of all changes at the end of the article.
%\listofchanges

\end{document}

% End of file `sample63.tex'.
